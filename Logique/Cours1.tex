\documentclass{article}
\usepackage{amsmath}
\usepackage{amssymb}
\usepackage[utf8]{inputenc}

\begin{document}

\section*{Résumé : Théorie de la preuve}

\subsection*{Définition (axiomatique à la Hilbert)}
Les schémas d'axiomes de la logique des prédicats incluent :
\begin{itemize}
    \item Les schémas d'axiomes de la logique propositionnelle.
    \item Les deux axiomes spécifiques :
    \[
    \forall x \, A \to (A)\{x \backslash t\}
    \]
    \[
    (A)\{x \backslash t\} \to \exists x \, A
    \]
    où $(A)\{x \backslash t\}$ désigne une substitution quelconque.
\end{itemize}
Les règles d'inférence sont :
\begin{enumerate}
    \item \textbf{Modus Ponens} :
    \[
    \frac{A \quad A \to B}{B}
    \]
    \item Les deux règles pour les quantificateurs :
    \[
    \frac{A \to B}{A \to \forall x \, B} \quad \text{(si $x$ n'apparaît pas libre dans $A$)}
    \]
    \[
    \frac{A \to B}{\exists x \, A \to B} \quad \text{(si $x$ n'apparaît pas libre dans $B$)}
    \]
\end{enumerate}

\subsection*{Preuve, prouvabilité, consistance}
\begin{itemize}
    \item Les définitions de preuve, prouvabilité et consistance sont similaires à celles de la logique propositionnelle.
    \item Définition de la déduction :
    \begin{itemize}
        \item Plus complexe qu'en logique propositionnelle.
        \item Les règles pour les quantificateurs ne s'appliquent qu'à des axiomes logiques.
    \end{itemize}
    \item La déduction peut être réduite à la validité par un théorème de déduction similaire à celui de la logique propositionnelle.
\end{itemize}

\subsection*{Propriétés importantes}
\begin{enumerate}
    \item \textbf{Théorème d'adéquation :} si $\vdash A$, alors $\vDash A$.
    \item \textbf{Théorème de complétude :} si $\vDash A$, alors $\vdash A$.
    \item \textbf{Théorème de semi-décidabilité :} 
    \begin{itemize}
        \item La logique des prédicats est semi-décidable.
        \item Il existe une procédure effective qui, pour toute formule $A$ :
        \begin{itemize}
            \item Si $A$ est valide, retourne "oui".
            \item Sinon, la procédure peut s'arrêter avec "non" ou ne jamais s'arrêter.
        \end{itemize}
    \end{itemize}
\end{enumerate}

\subsection*{Équivalences}
\subsubsection*{Équivalences propositionnelles}
\begin{itemize}
    \item Théorème de remplacement des équivalences.
    \item Élimination des connecteurs.
    \item Propriétés algébriques.
\end{itemize}

\subsubsection*{Équivalences relatives aux quantificateurs}
\begin{itemize}
    \item Fermeture universelle : si $A$ n'a pas d'occurrence libre de $x$, alors 
    \[
    \vdash A \iff \vdash \forall x \, A.
    \]
    \item Fermeture existentielle : si $A$ n'a pas d'occurrence libre de $x$, alors 
    \[
    \exists A \text{ est satisfiable } \iff \exists x \, A \text{ est satisfiable.}
    \]
    \item Quantification répétée : 
    \[
    \vdash (Q_1 x \, Q_2 x \, A) \iff Q_2 x \, A.
    \]
    \item Quantification sans variable libre : si $A$ n'a pas d'occurrence libre de $x$, alors 
    \[
    \vdash (Q x \, A) \iff A.
    \]
\end{itemize}

\subsection*{Formes normales}
\begin{itemize}
    \item Une formule $A$ peut être mise en forme normale prénexe, forme normale de Skolem, ou forme normale clausale.
    \item Les algorithmes garantissent l'équivalence ou la préservation de la satisfiabilité.
\end{itemize}

\end{document}
